\documentclass[10pt,letterpaper]{book}
\usepackage[latin1]{inputenc}
\usepackage[T1]{fontenc}
\usepackage{amsmath}
\usepackage{amsfonts}
\usepackage{amssymb}
\usepackage{makeidx}
\usepackage{graphicx}
\usepackage{listings}
\usepackage{hyperref}
\hypersetup{
    colorlinks=true,
    linkcolor=blue,
    filecolor=magenta,
    urlcolor=cyan,
}


\begin{document}
\title{VIM - Notes}
\tableofcontents
\chapter{Commands}
\section{Movements}
\paragraph{w}: Move one word forward
\paragraph{b}: Move one word backward
\paragraph{e}: Move to the end of the current or next word
\paragraph{\textdollar}: Move to end of line
\paragraph{0 or \^{}}: Move to beginning of line
\paragraph{G}: Move to end of file.
\paragraph{gg}: Move to beginning of file.
\paragraph{f-char}: Move right to the next occurence of \textit{char}.
\paragraph{t-char}: Move right before the next occurence of \textit{char}.
\paragraph{F-char}: Move left to the next occurence of \textit{char}.
\paragraph{T-char}: Move left before the next occurence of \textit{char}.
\paragraph{;}: typing \textit{;} will repeat the last command; it is very useful with commands used for moving around. For example: \textit{fo} will position the cursor to the next \textit{o} character; then hitting \textit{;} will move to the next occurence of \textit{o}.

\section{Misc}
\paragraph{:[range]g/pattern/cmd}: This acts on the specified \textit{[range]} (default whole file), by executing the Ex command \textit{cmd} for each line matching \textit{pattern} (an Ex command is one starting with a colon such as :d for delete). Example:\\
\textit{:g/pattern/d}: delete all lines matching \textit{pattern}\\
\paragraph{i}: Enter insert mode at the position of the cursor
\paragraph{I}: Enter insert mode at beginning of the line
\paragraph{<esc>}: Enter normal mode.
\paragraph{a}: Enter insert mode and append at one character after cursor's current position.
\paragraph{A}: Enter insert mode and append at end of the line.
\paragraph{R}: Enter replace mode (basically to overwrite stuff).
\paragraph{r}: Replace single character at cursor.

\paragraph{.}: The dot command lets us repeat the last command.
\paragraph{u}: Undo last change.
\paragraph{<C-a>, <C-x>}: perform addition and subtraction on numbers.
\paragraph{gU}: converts into upper case. Must be used with something else. For example \textit{gUap} converts an entire paragraph, \textit{gUaw} converts a word, etc.
\paragraph{gu}: converts into lower case. Must be used with something else. For example \textit{guap} converts an entire paragraph, \textit{guaw} converts a word, etc.
\paragraph{v}: Enter character-wise visual mode.
\paragraph{V}: Enter line-wise visual mode.
\paragraph{C-v}: Enter block-wise visual mode.

\paragraph{v j}: Enter visual mode and select current line.
\paragraph{v \$ }: Enter visual mode and select until end of current line.
\paragraph{v  \^{} or v 0}: Enter visual mode and select until beginning of current line.
\paragraph{v ip}: Enter visual mode and select everything between two empty lines in relation to current position of the cursor.

\paragraph{Select multiple lines and insert at the beginning of each line with multiple cursors}: go at the beginning of the first line, enter block-wise visual mode, select the other lines, then Shift-i and you get one cursor ready for entry at the beginning of each line.

\paragraph{operator-modifier}: \textit{o} is the operator modifier.\\
The effect of \textit{o} is that it re-selects the keyword under-cursor within the whole target area.\\
For example, \textit{cop} means use command c in this paragraph applied to all words equal to the word where the cursor is on. It can be incredibly useful to change the same word everywhere in a line/paragraph/document.
\paragraph{preset occurrence}: move onto the first word to change in several places and use \textit{go} to set the first keyword; then move to the second word to change and repeat the \textit{go} command. Now 2 presets have been established and with \textit{cp} I can change both words everywhere in the paragraph.

\paragraph{Select and change the same word in several places}: search for the word to change using the normal search (i.e. using /). Then use the command \textit{cgn} to cancel the word, type the new word and then use the key \textit{n} to move to the next occurence and then use the the \textit{'.'} to repeat the change.\\
The text-object \textit{gn} works like other text-objects with all commands. You can for example use it with \textit{d} to delete the matches.

\paragraph{Table of contents with links:} in order to have a table of content in the pdf file with links to the sections, use the package and the command as below:\\
\begin{lstlisting}
\usepackage{hyperref}
\hypersetup
    colorlinks=true,
    linkcolor=blue,
    filecolor=magenta,
    urlcolor=cyan,
}
\end{lstlisting}
\paragraph{Changing the size of the vim window:} To change programatically the size of the window use the commands below in the command line and/or in the .vimrc file.
\begin{lstlisting}
set lines=50 columns=100
\end{lstlisting}
\paragraph{Showing pieces of codes in a document:} To show pieces of code in a document use the \emph{lstlisting} environment. The package \emph{listings} must be added at the beginning of the document.
\section{Delete Stuff}
\paragraph{x}: The x command deletes the character under the cursor.
\paragraph{s}: Deletes character under the cursor and enters insert mode.
\paragraph{dw}: Delete the word the cursor is on
\paragraph{db}: Delete the word before the one the cursor is on
\paragraph{dd}: Delete entire line
\paragraph{d\$}: Delete from cursor to the end of current line.

\paragraph{dap}: Delete entire paragraph
\paragraph{c}: Delete the current selection and drop into insert mode.
\paragraph{C}: delete from cursor's position to the end of the line.

\paragraph{cw}: Delete to the end of the word and then drop into Insert mode
\paragraph{cb}: Delete from the beginning of the word and then drop into Insert mode
\paragraph{cc}: Delete entire line and then drop into insert mode
\paragraph{cap}: Delete entire paragraph and then drop into Insert mode
\\ \\
\textbf{NOTE}: all the c and d commands can be used with a number n before, which means the command is repeated n times. For example: \textit{3dd} will delete 3 lines, \textit{3dw} will delete 3 words, etc.
\paragraph{<C-h>}: from Insert Mode, delete back 1 character
\paragraph{<C-w>}: from Insert Mode, delete back 1 word
\paragraph{<C-u>}: from Insert Mode, delete back to start of line
\section{Indentation}
\paragraph{>G}:The >G command increases the indentation from the current line until the end of the file.
\paragraph{>>}: Intent current line
\paragraph{=}: Autoindent
\section{Search}
\paragraph{In normal mode type \textbackslash}: this will open the command line. Then enter the word to search. At the end of the word add the \textbackslash c option to run a case-insensitive search.

\paragraph{f\{char\}}: Look ahead for the next occurrence of the specified character and then move the cursor directly to it if a match is found.
\paragraph{F\{char\}}: Look backwards for the next occurrence of the specified character and then move the cursor directly to it if a match is found.
\paragraph{t\{char\}}: Forward to the character before the next occurrence of \{char\}
\paragraph{T\{char\}}: Backward to the character before the next occurrence of \{char\}
\paragraph{;}: repeat the last search that the \textit{f\{char\}} (or \textit{F\{char\}}) command performed
\paragraph{,}: repeat the last \textit{f\{char\}} (or \textit{F\{char\}} or \textit{t\{char\}} or \textit{T\{char\}})search in the reverse direction
\paragraph{/pattern}: scan document for next match
\paragraph{?pattern}: scan document for previous match
\paragraph{n}: repeat \textit{/pattern} (o \textit{?pattern})
\paragraph{N}: reverse \textit{/pattern} (o \textit{?pattern})
\paragraph{*}: search for word under the cursor (n and N work with * too)
\section{Copy and Paste}
\paragraph{yy}: copy a line into register
\paragraph{yw}: copy the word the cursor is on into register
\paragraph{yb}: copy the word before the one the cursor is on into register\\ \\
\textbf{NOTE}: all the y commands can be used with a number n before, which means the command is repeated n times. For example: \textit{3yy} will copy 3 lines, \textit{3yw} will copy 3 words, etc.
\paragraph{p}: paste register
\paragraph{<C-r>0}: paste register 0 from Insert Mode
\section{Multiple Files}
\paragraph{:ls}: Buffers list. The one with the \% symbol is the active one
\paragraph{<C-\^{}>}: toggle between files.
\paragraph{:bN}: where N is the buffer number, switch to buffer N.
\paragraph{<C-w>}w: Cycle between open windows
\paragraph{<C-w>s}: split window horizontally.
\paragraph{<C-w>v}: split window vertically.
\paragraph{<C-w>c}: close current window.
\paragraph{<C-w>o}: close all windows except the active one.
\paragraph{:sp {file}}: split window horizontally and open {file}.
\paragraph{:vsp {file}}: split window vertically and open {file}.
\paragraph{:tabe {filename}}: open {filename} in a new tab.
\paragraph{<C-w>T}: move the current window into its own tab.
\paragraph{:tabc}: close the current tab page and all of its windows.
\paragraph{:tabo}: keep the active tab page, closing all others.
\paragraph{:tabn N}: switch to tab number N.
\paragraph{:tabn}: switch to next tab.
\paragraph{:tabp}: switch to previous tab.
\paragraph{:tabmove N}: move tab to position N (0 is the beginning).
\section{Files}
\paragraph{:edit or :e}: opens a file either by specifying
an absolute or a relative filepath
\paragraph{:find}: allows to open a file by its name without having to provide a fully qualified path
\paragraph{Explore File System}: If we launch Vim with the path to a directory (for example \textit{vim .}) rather than a file, it will start up with a file explorer window
\paragraph{:Explore or :E}: opens the file explorer
\paragraph{:edit \{path\}}: opens the file explorer
\paragraph{:Vexplore}: opens the file explorer in a vertical split window
\paragraph{\textit{buffer number}:bdelete or :bd}: deletes the buffer(s) specified (it can be a range or a list too.
\section{Links}
\href{https://github.com/t9md/atom-vim-mode-plus/wiki/AdvancedTopicTutorial#0-text-to-use-in-this-tour}{Some advanced topics}\\

\chapter{Vim Latex}
Vim-Latex is a vim add-on/plug-in which is used to make Latex files in Vim.\\
\href{http://vim-latex.sourceforge.net/documentation/latex-suite/}{The manual is here}.\\
Check the chapter 3 which is about the macros available to quickly insert the different elements of a Latex document.
\section{Folding}
\paragraph{\textbackslash rf}: This will fold up the entire file. Does not seem to work in Atom.
\paragraph{\textbackslash za}: Open and close folds.
\paragraph{C-k C-number}: where \textit{number} is a number from 1 to 9. Folds/unfolds the level specified by \textit{number}.


\end{document}
